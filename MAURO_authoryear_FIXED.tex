\documentclass[12pt]{article}

% ======= PAQUETES =======
\usepackage[utf8]{inputenc}
\usepackage[T1]{fontenc}
\usepackage[spanish]{babel}
\usepackage{csquotes}
\usepackage{natbib}
\usepackage{url}
\usepackage{amsmath, amssymb}

% ======= METADATOS =======
\title{Título del Documento}
\author{Autor del Documento}
\date{\today}

\begin{document}

\maketitle

Este es un ejemplo corregido del archivo original. Aquí puedes comenzar a redactar tu contenido.

La semiótica social, en particular la desarrollada por \citet{Halliday2004}, se ha convertido en una herramienta clave para analizar las prácticas discursivas en diversos contextos socioculturales.

La noción de \textit{metáfora visual}, explorada ampliamente por \citet{Forceville1996}, permite entender cómo las imágenes comunican significados complejos más allá del lenguaje verbal.

\section{Marco Teórico}

El análisis de los recursos semióticos según \citet{KressVanLeeuwen2006} permite observar cómo se construyen significados en diferentes modalidades comunicativas.

\section{Discusión}

El diseño multimodal no sólo considera los elementos visuales, sino también la disposición espacial, la tipografía y otros factores que influyen en la interpretación del mensaje.

\section{Conclusiones}

Los estudios de semiótica visual permiten comprender cómo los significados no lingüísticos forman parte integral de la comunicación contemporánea.

% ======= BIBLIOGRAFÍA =======
\bibliographystyle{apalike}
\bibliography{referencias}

\end{document}