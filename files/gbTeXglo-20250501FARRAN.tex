
\newglossaryentry{@glo232-ciecs}{
type = \acronymtype,
name         = {CIECS},
description  = {Centro de Investigaciones y Estudios sobre la Cultura y la Sociedad},
first        = {Centro de investigaciones y Estudios sobre la Cultura y la Sociedad (CIECS)},
text         = {CIECS},
}
\newglossaryentry{@glo233-conicet}{
type = \acronymtype,
name         = {CONICET},
description  = {Consejo Nacional de Investigaciones Científicas y Técnicas},
first        = {Consejo Nacional de Investigaciones Científicas y Técnicas (CONICET)},
text         = {CONICET},
}
\newglossaryentry{@glo234-serquaser}{
name         = {Ser qua ser},
description  = {"Ser qua ser" es una frase latina que significa "el ser en cuanto ser" o "el ser como tal". En filosofía, se refiere al estudio del ser en sí mismo, independientemente de sus cualidades o atributos particulares.},
text         = {ser qua ser},
}
\newglossaryentry{@glo235-unc}{
type = \acronymtype,
name         = {UNC},
description  = {Universidad Nacional de Córdoba},
first        = {Universidad Nacional de Córdoba (UNC)},
text         = {UNC},
}
\newglossaryentry{@glo236-esencialismo}{
name         = {Esencialismo},
description  = {Doctrina que sostiene que los individuos o grupos poseen una esencia fija y universal que determina su identidad, carácter o comportamiento, independientemente del contexto social o histórico.},
text         = {esencialismo},
}
\newglossaryentry{@glo237-constructivismo}{
name         = {Constructivismo},
description  = {Enfoque teórico que afirma que las identidades, conocimientos o categorías sociales no son naturales ni fijas, sino que se construyen históricamente a través de procesos culturales, discurisvos y sociales.},
text         = {constructivismo},
}