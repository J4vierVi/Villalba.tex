\ifPDF
\chapter[Autores]{Autores}
\Author{Autores}
\setcounter{PrimPag}{\theCurrentPage}
	\else
	\ifHTMLEPUB
	\chapter{Autores}
	\fi
\fi

\paragraph{Gala Aznarez Carini}

Licenciada en Psicología por la \gls{@glo235-unc}. Actualmente sus trabajos articulan el psicoanálisis lacaniano con el pensamiento político contemporáneo para pensar los diversos modos de subjetivación política según las especificidades de su localización. Mail: \url{gala_az@hotmail.com}

\paragraph{Emmanuel Biset}

Licenciado en Filosofía y Licenciado en Ciencia Política por la Universidad Nacional de Río Cuarto. Doctor en Filosofía por la \gls{@glo235-unc} y por la Université Paris 8. Tesis doctoral: \enquote{Violencia, justicia y política. Una lectura de Jacques Derrida}. Investigador Asistente de \gls{@glo233-conicet} sobre el problema de la justicia en el pensamiento político postfundacional. Mail: \url{bisetico@hotmail.com}

\paragraph{Andrés Daín}

Licenciado en Ciencia Política. Profesor universitario. Doctorando en Ciencia Política. Becario de \gls{@glo233-conicet}. Su investigación se titula \enquote{Análisis político-ideológico de las urbanizaciones privadas en Argentina}. Mail: \url{andresdain@gmail.com}

\paragraph{Roque Farrán}

Licenciado en Psicología por la \gls{@glo235-unc}. Doctorando en Filosofía por la \gls{@glo235-unc}. Becario Doctoral de \gls{@glo233-conicet}. Su investigación se titula \enquote{El concepto de sujeto en Alain Badiou y Jacques Lacan. Dimensiones ontológicas y políticas} Mail: \url{roquefarran@gmail.com}

\paragraph{Daniel Groisman}

Licenciado en Ciencia Política por la Universidad Católica de Córdoba. Maestrando en \enquote{Estudios Interdisciplinarios de la Subjetividad} por la Universidad de Buenos Aires. Doctorando en filosofía por la \gls{@glo235-unc}. Su investigación aborda la temática del sujeto y la verdad en y a través de la obra de Alain Badiou.~Becario~de la~Secretaría de Ciencia y Tecnología de la \gls{@glo235-unc}. Mail: \url{danielgroisman@gmail.com}

\paragraph{Manuel Moyano}

Licenciado en Ciencia Política por la Universidad Católica de Córdoba. Investiga el problema de la experiencia en el pensamiento político de Giorgio Agamben. Mail: \url{manumoyano@gmail.com}

\paragraph{Juan Manuel Reynares}

Licenciado en Ciencia Política por la Universidad Nacional de Villa María. Actualmente es becario doctoral \gls{@glo233-conicet}, con sede en el Centro de Estudios Avanzados de la \gls{@glo235-unc}. Está investigando la constitución de identidades políticas en la provincia de Córdoba desde el retorno a la democracia. Mail: \url{juanmanuelreynares@hotmail.com}

\paragraph{María Aurora Romero}

Licenciada en Sociología por la Universidad Empresarial Siglo 21. Maestranda en Sociología en el Centro de Estudios Avanzados de la \gls{@glo235-unc}. Doctoranda en Ciencias Sociales en la Universidad de Buenos Aires.~Becaria Doctoral~de \gls{@glo233-conicet}. Su investigación se titula \enquote{Entre el saber y el poder: el campo científico. Una perspectiva crítica al paradigma cientificista}. Mail: \url{maauroraromero@gmail.com}

\paragraph{Mercedes Vargas}

Licenciada en Psicología por la \gls{@glo235-unc}. Actualmente becaria doctoral FONCYT en el Centro de Investigaciones y Estudios en Cultura y Sociedad (\gls{@glo232-ciecs}-UE \gls{@glo233-conicet}-\gls{@glo235-unc}) en el marco de un proyecto PICT que se propone indagar la constitución de identidades políticas durante el primer peronismo desde una mirada \enquote{desde abajo}. Sus trabajos intentan incorporar el psicoanálisis lacaniano al análisis político para pensar los procesos de subjetivación política. Mail: \url{mer_chan86@hotmail.com}

\ifPDF
\separata{Autores}
\fi

