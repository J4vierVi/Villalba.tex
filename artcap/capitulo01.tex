\ifPDF
\chapter[\textbf{Emmanuel Biset}\\ Ontología de la diferencia]{Ontología de la diferencia}
\Author{Emmanuel Biset}
\chaptermark{Ontología de la diferencia}
% encabezado para autor
\begin{center}
	\nombreautor{Emmanuel Biset} \\[1ex]
	\makebox[2pc]{\dotfill}
	\end{center}
\setcounter{PrimPag}{\theCurrentPage}
	\else
	\ifHTMLEPUB
	\chapter{Ontología de la diferencia}
	\fi
\fi


\epigraph{\emph{\enquote{La diferencia entre Deleuze y Derrida como diferencia propia \rdm{y en consecuencia, como identidad en sí dividida} de un tiempo, de un presente de pensamiento que habrá formado una inflexión decisiva, sigue estando por pensar}.}}{Jean-Luc Nancy}

\section{Introducción}

Existen ciertos conceptos y no otros que marcan determinada época. Conceptos que al mismo tiempo que constituyen un índice del tiempo vivido intervienen en él. Uno de esos conceptos es el de \enquote{diferencia}, concepto que se ha constituido como un indicio central para pensar el mundo contemporáneo en, por lo menos, dos sentidos. De un lado, el término diferencia ha servido para designar toda una corriente del pensamiento contemporáneo: aquello que se ha denominado \enquote{filosofía de la diferencia}. De otro lado, el término diferencia en un sentido político viene a designar una época en la cual se han pluralizado las formas de vida y la alteridad se ha convertido en un problema ineludible. Un pensamiento de la diferencia en política parece remitir a la pluralidad expandida del mundo contemporáneo y así a las discusiones sobre multiculturalismo, interculturalidad, etc., y al problema de la alteridad radical, y así a las discusiones sobre migrantes, sexualidades, pueblos originarios, etc.

Aquí no se indaga la supuesta pluralidad contemporánea, sino en la discusión de diferentes pensamientos de la diferencia se da cuenta del estatuto ontológico de la diferencia. Cuando nos referimos a estatuto ontológico de la diferencia señalamos que no se trata de pensar la diferencia entre elementos de una realidad ya constituida, sino de la diferencia como constituyente\index[concepto]{Diferencia!como constituyente} de esa misma realidad. Este mismo problema surge en el plano de la significación. Si queremos precisar el significado de un término como \enquote{diferencia} lo reducimos al plano de un signo entre otros, siendo que la diferencia menta el mismo proceso de significación. Por lo que el primer objetivo del texto es mostrar cómo se ha roto con aquellos planteos que hacen de la diferencia algo secundario respecto a elementos constituidos. Para cumplir este objetivo se avanza en tres pasos.

En un primer momento se muestra el contexto en el cual la diferencia dejó de ser un concepto entre otros para caracterizar a toda una generación de autores. De este contexto destacamos el enfrentamiento con la dialéctica hegeliana, la elaboración de una filosofía de la diferencia en Heidegger y el lugar de la diferencia lingüística en el estructuralismo. Planteos que no sólo se muestran como indicios de época, sino que articulan una serie de supuestos desde los que se elabora el pensamiento de la diferencia.

En un segundo momento, se analiza la diferencia en los pensamientos de Gilles Deleuze y Jacques Derrida. Si estos autores son identificados como los autores de la diferencia, el objetivo del texto es indicar qué sentido adquiere la diferencia en cada uno, no sólo mostrando sus antecedentes, sino aquellos aspectos que los distancian. Será la referencia a Hegel aquello que permita notar dos filosofías de la diferencia.

En un tercer momento, se muestra el carácter político de una ontología de la diferencia.\footcite[\emph{Ontología diferencial} es el título de una propuesta reciente de Miguel de Beistegui que cruza los planteos de Heidegger y Deleuze para elaborar una filosofía que escape a su fragmentación. Aquí, a diferencia de Beistegui, se trata de pensar desde una perspectiva política esta ontología diferencial. Véase][]{@6960-BEISTEGUI2004} Para ello, se dan argumentos en dos sentidos: se analiza el carácter \enquote{constituyente} de la diferencia a partir de la distinción entre lo óntico, lo ontológico y lo trascendental, y se propone, a distancia de ciertos planteos contemporáneos, la politicidad de la diferencia como procesos de temporalización y espaciamiento en tanto condiciones de existencia de lo dado. Este apartado, indicativo ante todo, pretende ser el aporte del texto.

En resumidas cuentas, efectuamos un recorrido singular para mostrar el carácter político de la diferencia. Lo cual nos permite dar cuenta de una particular vinculación teórica entre ontología y política.

\section{Dialéctica y diferencia}

Al comenzar \emph{Diferencia y repetición}, Gilles Deleuze escribe sobre el contexto en que surge el problema de la diferencia:

\begin{quote}
	El tema aquí tratado se encuentra, sin duda alguna, en la atmósfera de nuestro tiempo. Sus signos pueden ser detectados: la orientación cada vez más acentuada de Heidegger hacia una filosofía de la Diferencia ontológica; el ejercicio del estructuralismo, basado en una distribución de caracteres diferenciales en un espacio de coexistencia; el arte de la novela contemporánea, que gira en torno de la diferencia y de la repetición, no sólo en su reflexión más abstracta sino también en sus técnicas efectivas; el descubrimiento, en toda clase de campos, de un poder propio de repetición, que sería tanto la del inconsciente como la del lenguaje y del arte. Todos estos signos pueden ser atribuidos a un anti-hegelianismo generalizado: la diferencia y la repetición ocuparon el lugar de lo idéntico y de lo negativo, de la identidad y de la contradicción.\footcite[15]{@6961-DELEUZE2002}
\end{quote}

Desde la perspectiva de Deleuze, el pensamiento de la diferencia puede ser entendido desde un antihegelianismo generalizado, es decir, se comprende como reacción ante la hegemonía de la dialéctica. Si bien es posible indicar que existen otras corrientes con una presencia ineludible \rdm{así la fenomenología}, la noción de diferencia surge en oposición a cierta interpretación de Hegel. Esto resulta central, pues en gran medida la historia intelectual del pensamiento francés contemporáneo comienza con la lectura que realiza Alexander Kojève de la fenomenología hegeliana

\begin{quote}
	Si existe un signo del cambio de mentalidades \rdm{rebelión contra el neo-kantismo, eclipse del bergsonismo}, desde luego que es la vuelta firme de Hegel. Éste, proscrito por los neokantianos, de repente se vuelve, curiosamente, un autor de vanguardia citado con respeto en los círculos más avanzados. Este renacimiento parece deberse a dos razones principales. Una es el nuevo período de interés hacia el marxismo, tras la revolución rusa.(\dots) La otra razón es la influencia del curso pronunciado por Alexandre Kojève en la Escuela Práctica de Altos Estudios a partir de 1933 y que se prolongará hasta 1939. \footcite[28]{@6962-DESCOMBES1998}
\end{quote}

La relación con Kojève es central porque permite entender cómo se lee a Hegel en el pensamiento francés contemporáneo. Una interpretación en la cual se recupera la \emph{Fenomenología del espíritu}, apartándose de las lecturas que ven en el filósofo alemán sólo la hipóstasis de la razón. Esta lectura acentúa el origen irrazonable de lo razonable en relación con las filosofías de la existencia de principios de siglo, es decir, en proximidad con Husserl y Heidegger. Ante las lecturas panlogistas que destacan la identificación de lo real y lo racional, en el Hegel de Kojève el pensamiento es el movimiento de la razón hacia su otro, y por ello es una ampliación de la razón. Se trata de la \emph{negatividad} pensada en términos antropológicos, o mejor, de una antropología ontológica donde el hombre es el motor de la historia: \enquote{\ldots antropológico en el sentido que se trata ahí de \enquote{existencia}, es decir, de deseo y de \emph{acción}. Hegel no es simplemente un intelectualista: sin la creación por la acción negadora no hay contemplación de lo dado}.\footcite[53]{@6963-KOJEVE1996} En esta perspectiva, la dialéctica expresa un humanismo, esto es, la dialéctica es ontológicamente humanista porque todo lo que tiene sentido se decide en la historia humana entendida como acción transformadora del hombre.

La lectura de Kojève será central al trabajar Hegel a la luz de la filosofía contemporánea, marcando así a toda una generación del pensamiento francés entre quienes se puede citar a Georges Bataille, Raymond Aron, Alexandre Koyré, Pierre Klossowski, Jacques Lacan, Maurice Merleau-Ponty, Eric Weil. La dialéctica goza de un importante prestigio en la Francia posterior a 1930. Quizá el mayor indicio de esta relevancia se encuentra en \emph{La crítica de la razón dialéctica} de Jean Paul Sartre, publicada en 1960, y donde el marxismo como razón dialéctica constituye el \enquote{horizonte irrebasable} de nuestro tiempo: \enquote{Nuestro tiempo será, pues, \emph{crítico} porque tratará de determinar la validez y los límites de la Razón dialéctica, lo que supone indicar las oposiciones y los lazos de esta Razón con la Razón analítica y positivista}. \footcite[11]{@6964-SARTRE1995}

Frente a esta generación, ciertos autores que comienzan a escribir en las décadas del 50 y 60 se definen por su crítica a la dialéctica. Si bien existen diferencias de estilo, de acento, de interpretación, se comparte esta oposición al hegelianismo. Autores como Michel Foucault, Gilles Deleuze, Jacques Derrida, se caracterizan por la ruptura con Hegel. Señala Michel Foucault: \enquote{(\dots) toda nuestra época, bien sea por la lógica o por la epistemología, bien sea por Marx o por Nietzsche, intenta escapar de Hegel}. \footcite[59]{@6984-FOUCAULT1973} Esta ruptura con Hegel resulta central para pensar el contexto de emergencia del pensamiento de la diferencia. Pues vale recordar que la dialéctica hegeliana se construye como una lógica donde la alteridad tiene un lugar constitutivo. La negatividad nombra un movimiento de alienación, de un hacerse otro. La diferencia como \emph{mediación} es constitutiva del movimiento dialéctico. Al mismo tiempo, la identidad juega un rol central, pues la mediación se produce en vistas a una reconciliación final. En este marco es necesario destacar que la crítica a Hegel surge de la oposición entre identidad y diferencia, pues la cuestión es de qué modo la predominancia de la identidad reduciría la diferencia, es decir, la alteridad a la mismidad. Desde esta perspectiva, la dialéctica al considerar la diferencia como negación produciría un sometimiento de lo otro en una articulación superior, en una unidad idéntica superior jerárquicamente.

En este contexto de crítica a la dialéctica es posible ubicar dos de los indicios de época que Deleuze señala en su cita: la filosofía de la diferencia de Martin Heidegger y el estructuralismo de Ferdinand de Saussure. En Heidegger aparece explícitamente la confrontación entre dialéctica y diferencia en un escrito central para la época: \emph{Identidad y diferencia}. Publicado en el año 1957, el texto se compone de dos conferencias: \enquote{El principio de identidad} y \enquote{La constitución onto-teo-lógica de la metafísica}. Si la diferencia ontológica es constitutiva del pensamiento heideggeriano, es su radicalización en una filosofía de la diferencia aquello que constituye un indicio central para el pensamiento francés de la década del 60. En el segundo de los artículos citados, texto de 1957 escrito como finalización de un curso sobre Hegel, Heidegger nomina su pensamiento desde la noción de diferencia oponiéndola a la dialéctica hegeliana:

\begin{quote}
	Para Hegel, el asunto del pensar es el ser en relación con lo que fue pensado sobre lo ente en el pensar absoluto y en cuanto tal. Para nosotros, el asunto del pensar es lo mismo, y por lo tanto, el ser, pero el ser desde la perspectiva de su diferencia con lo ente. Digámoslo con más precisión todavía: para Hegel, el asunto del pensar es el pensamiento como concepto absoluto. Para nosotros, el asunto del pensar \rdm{usando un nombre provisional}, es la diferencia \emph{en cuanto} diferencia. \footcite[107]{@6965-HEIDEGGER1988}
\end{quote}

La noción de diferencia retomando la diferencia ontológica sirve aquí para nombrar la distancia con Hegel. De modo que en Heidegger encontramos claramente la oposición entre pensamiento de la diferencia y filosofía dialéctica. Si esta oposición constituye el primer aspecto a destacar contextualmente, el segundo aspecto es la necesidad de pensar la diferencia en cuanto diferencia. Esto significa: pensar la diferencia en cuanto tal\index[concepto]{Diferencia!en cuanto tal}. Este \enquote{en cuanto tal} permite notar dos cosas respecto al tratamiento de la diferencia: primero, que con el término diferencia no se nombra la diferencia de cosas existentes en el mundo: la diferencia óntica\index[concepto]{Diferencia!óntica}; segundo, que tampoco el término diferencia se dirige a una distinción establecida por el entendimiento. Heidegger nos ayuda a comprender que con este término no se está aludiendo a dos cosas distintas en el mundo ni a una distinción establecida por el entendimiento humano. Estas distinciones son secundarias puesto que suponen, ante todo, la identidad de aquellas cosas que luego entrarán en una relación diferencial. En cualquiera de estos casos la diferencia es sobreañadida a una realidad preexistente, sea como representación del entendimiento, sea como vinculación en la experiencia. Por lo que la diferencia se reduce a una diferencia entre entes.

Heidegger señala que la cuestión es pensar la diferencia entre ente y ser. Esta diferencia no es posterior a dos realidades llamadas por caso ser y ente, sino que ser y ente aparecen a partir de la diferencia. Esto resulta central, pues la diferencia no es posterior, sino es la misma \emph{posibilidad} del ser y el ente, por lo que existe una \enquote{primacía ontológica} de la diferencia sobre el ser y el ente. No se puede distinguir ser y ente como dos entes singulares, pues en tal caso se elimina la pregunta por la diferencia ontológica. Se trata de pensar en el ente mismo el ser, por lo que ser y ente no son algo distinto, son lo mismo, o mejor, es la diferencia en lo mismo: \enquote{Ser y ente son lo Mismo; sólo en la diferenciación entre ser y ente está unido de propio lo Mismo (el ser del ente, el ente en su ser) en la unidad con él mismo. Ser no es algo distinto de lo ente; si fuera algo distinto sería, una vez más, ente, y la Diferencia ontológica quedaría invertida en mera Diferencia óntica}. \footcites[176]{@6966-POGGELER1993}[Si ya tempranamente la cuestión de la diferencia ontológica resulta central, será su radicalización como pensamiento de la diferencia aquello que constituya un marco de referencia para la filosofía francesa. Radicalización en cuanto se trata de pensar la diferencia en cuanto diferencia. Pensamiento de la diferencia en cuanto tal que se enfrenta a una dificultad constitutiva debido a que existen diversos términos en alemán que el autor utiliza para nombrarla. Véase][]{@6990-RESTA1988} Esta será la principal crítica a la tradición, convertir al ser en un ente más, incluso un ente supremo que fundaría el resto de los entes (onto-teo-logía). Pero si el ser se piensa como un ente entre otros, olvidamos la pregunta por la diferencia ontológica. De modo que el pensamiento de Heidegger resulta central porque, de un lado, aparece una \enquote{filosofía de la diferencia} en expresa confrontación con Hegel y, de otro lado, un pensamiento de la diferencia en cuanto tal \index[concepto]{Diferencia!en cuanto tal} lleva a la formulación de la diferencia ontológica, es decir, a mostrar el estatuto ontológico de la diferencia. En otros términos, la diferencia no es posterior a la existencia de cosas en el mundo, sino que es aquello que hace posible al mundo como tal.

El otro indicio ineludible del contexto es el estructuralismo. \footcite[Francois Dosse en su historia intelectual del estructuralismo señala la relevancia para la década del 60 del cruce entre el programa nietzscheano-heideggeriano y el estructuralismo: \enquote{La búsqueda heideggeriana del logos se une aquí con la genealogía nietzscheana, y ambas encontraron en el estructuralismo un magnifico destino. La crítica del etnocentrismo, del eurocentrismo, van a acentuarse en los años cincuenta y sesenta con la marea estructuralista, que va a retomar el paradigma crítico del nietzscheao-heideggerismo}][416]{@6967-DOSSE2004} No se trata de una teoría entre otras, sino del marco teórico que hegemoniza la época. Una \enquote{aventura de la mirada} que pretende atravesar diversos campos proponiendo una forma singular de indagar todo objeto. En el estructuralismo la noción de diferencia no es una más, sino que puede considerarse su núcleo duro. Como es sabido, Saussure inaugura la ciencia del lenguaje en cuanto define la naturaleza del objeto y los métodos propios para su análisis. El objeto de la lingüística es la lengua como hecho social que forma el lenguaje en oposición a su manifestación individual, el habla. La lengua es un sistema de signos donde resulta constitutiva la diferencia:

\begin{quote}
	Todo lo precedente viene a decir que \emph{en la lengua no hay más que diferencias.} Todavía más: una diferencia supone, en general, términos positivos entre los cuales se establece; pero en la lengua \emph{sólo hay diferencias sin términos positivos.} Ya se considere el significante, ya el significado, la lengua no comporta ni ideas ni sonidos preexistentes al sistema lingüístico, sino solamente diferencias conceptuales y diferencias fónicas resultantes de ese sistema. \footcite[144]{@6968-SAUSSURE1945}
\end{quote}

La lengua introduce un principio de clasificación de los fenómenos del lenguaje. Esta clasificación se sustenta en la idea de signo que organiza la lingüística saussureana: el signo es una unidad discreta que se define por su combinatoria. El signo, como unidad fundamental de la lengua, une un significado y un significante, en términos de Saussure, una imagen acústica y un concepto. Signo que posee dos caracteres primordiales: en primer lugar, es arbitrario, pues el lazo que une el significante al significado es inmotivado; en segundo lugar, el significante posee un carácter lineal, es decir, se desarrolla en el tiempo, el significante posee una extensión que es mensurable en una línea de tiempo. Luego de establecer estos dos principios, es fundamental notar que el valor del signo surge de una relación diferencial. Esto significa que la lengua es un sistema donde todos los términos son solidarios y el valor de cada uno surge de la presencia simultánea de los otros: \enquote{En todos estos casos, pues, sorprendemos, en lugar de \emph{ideas} dadas de antemano, valores que emanan del sistema. Cuando se dice que los valores corresponden a conceptos, se sobreentiende que son puramente diferenciales, definidos no positivamente por su contenido, sino negativamente por sus relaciones con los otros términos del sistema. Su más exacta característica es la de ser lo que los otros no son}.\footcite[141]{@6968-SAUSSURE1945} Por ello señala Saussure que arbitrario y diferencial son dos cualidades correlativas que constituyen la lengua donde sólo hay diferencias sin términos positivos.

Siguiendo los señalamientos de Saussure, la diferencia será central en el estructuralismo puesto que no es un pensamiento de la relación entre signos ya constituidos, sino que el mismo valor o identidad del signo surge de la relación diferencial. En este sentido, la diferencia es \enquote{constitutiva} de los signos, pues un sistema diferencial precede y posibilita la identidad de los elementos. Por esto, en el estructuralismo se vinculan de modo inherente génesis y estructura: la estructura como sistema de diferencias muestra la génesis diferencial de todo elemento. Partiendo de esta perspectiva, el estructuralismo excedió su primera formulación restringida al campo lingüístico, para constituirse en un modo de abordar diferentes objetos donde posibilidad de determinar la significación surgía de consideraciones formales: \enquote{Las investigaciones estructurales carecerían de interés si las estructuras no fueran traducibles a modelos cuyas propiedades formales son comparables, con independencia de los elementos que las componen}.\footcite[256]{@6969-LEVISTRAUSS1977} Un ordenamiento es estructurado si es un sistema con cohesión interna que se revela en el estudio de las transformaciones del mismo. Un modelo geométrico de la diferencia constituye el sentido mismo, es decir, un modelo espacial construye la estructura como la forma desde la cual se analiza un objeto determinado. La estructura como imagen espacial sólo es posible a partir de la simultaneidad, es decir, del orden de la co-existencia. Resulta central así la simultaneidad de la forma. El estructuralismo vive de la constitución de totalidades coexistentes donde se organiza el sentido de un modo geométrico. La espacialidad constituye la idea de estructura como sistema donde la modificación de cualquier elemento entraña la modificación de todos los demás.

Sea en el pensamiento de Heidegger, sea en el estructuralismo como modo de abordar diversos objetos, la diferencia deja de ser considerada un significante entre otros para pasar a ser constitutivo de ambos pensamientos. Si bien sería posible desarrollar en extensión la diferencia en ambos marcos, aquí nos sirve para indicar no sólo la relevancia de la diferencia en un contexto determinado, sino en qué sentido adquiere un estatuto ontológico. Se trata, en última instancia, de pensar la diferencia en sí misma, lo que significa romper con una concepción relacional de la diferencia como vinculación de elementos previamente constituidos. La diferencia es ontológica en cuanto constituye los mismos elementos, esto es, la diferencia es \emph{genética}, sea en la precedencia respecto del ser y el ente en Heidegger, sea en la constitución del valor del signo en Saussure. Por esto, la pregunta es qué se entiende por \enquote{constitución} y cómo es posible vincular esto con la política. Para avanzar en este sentido, las referencias a Gilles Deleuze y Jacques Derrida resultan ineludibles.

\ifPDF
\begin{table}[!ht]
	\sf\footnotesize\setlength\tabcolsep{4pt}
	\centering
	\begin{tabular}{l | >{\raggedright\arraybackslash}m{3.8cm} | >{\raggedright\arraybackslash}m{4cm}}
		\toprule
		\textbf{Autor} & \textbf{Relación con la diferencia} & \textbf{Relación con la dialéctica} \\
		\midrule
		Heidegger &
		Piensa la diferencia como diferencia ontológica (entre ser y ente), no como distinción entre entes. Propone una filosofía de la diferencia. &
		Rechaza la dialéctica hegeliana. La diferencia no es mediación sino lo que hace posible el ser mismo. \\
		\midrule
		Saussure &
		La lengua está compuesta únicamente por diferencias sin términos positivos. El significado surge de relaciones diferenciales. &
		No recurre a la dialéctica. Sustituye el devenir por estructuras. La diferencia organiza el sentido sin síntesis. \\
		\midrule
		Hegel &
		Destaca la negatividad como deseo y acción. La diferencia es el motor del pensamiento como movimiento hacia lo otro. &
		Reformula la dialéctica como antropología ontológica. Luego será criticada por Deleuze, Foucault y Derrida. \\
		\bottomrule
	\end{tabular}
	\caption{Diferencias frente a la dialéctica hegeliana}
	\label{tab:Cuadro 1.1}
\end{table}
	\else
	\ifHTMLEPUB
	\begin{table}[!htbp]
		\sf\footnotesize\setlength\tabcolsep{4pt}
		\centering
		\begin{tabular}{l | >{\raggedright\arraybackslash}m{3.8cm} | >{\raggedright\arraybackslash}m{4cm}}
			\toprule
			\textbf{Autor} & \textbf{Relación con la diferencia} & \textbf{Relación con la dialéctica} \\
			\midrule
			Heidegger &
			Piensa la diferencia como diferencia ontológica (entre ser y ente), no como distinción entre entes. Propone una filosofía de la diferencia. &
			Rechaza la dialéctica hegeliana. La diferencia no es mediación sino lo que hace posible el ser mismo. \\
			\midrule
			Saussure &
			La lengua está compuesta únicamente por diferencias sin términos positivos. El significado surge de relaciones diferenciales. &
			No recurre a la dialéctica. Sustituye el devenir por estructuras. La diferencia organiza el sentido sin síntesis. \\
			\midrule
			Hegel &
			Destaca la negatividad como deseo y acción. La diferencia es el motor del pensamiento como movimiento hacia lo otro. &
			Reformula la dialéctica como antropología ontológica. Luego será criticada por Deleuze, Foucault y Derrida. \\
			\bottomrule
		\end{tabular}
		\caption{Diferencias frente a la dialéctica hegeliana}
		\label{tab:Cuadro 1.1}
	\end{table}
	\fi
\fi

\section{La diferencia afirmativa: Gilles Deleuze}

Desde los señalamientos contextuales, resulta posible indicar que la noción de diferencia va a constituir un índice y un factor en la Francia de los 60. Una de las figuras centrales en el esbozo de un pensamiento de la diferencia es Gilles Deleuze. En cierto sentido la categoría de diferencia se identifica con toda la obra deleuzeana, de ahí la dificultad de precisar un sentido. Ya en su escrito temprano sobre David Hume, de 1953, aparece la idea de un \enquote{principio de diferencia}. Cuando se pregunta qué es lo dado para el empirismo, indica que se trata de un conjunto de percepciones \rdm{un movimiento}, sin una identidad previa. El empirismo parte de la experiencia como una sucesión móvil de percepciones distintas. Por lo que el principio del empirismo no es aquel según el cual toda idea deriva de una impresión sensible, sino aquel según el cual todo lo que es separable es discernible y así todo lo discernible es diferente. El principio del empirismo es la diferencia.\footcite{@6970-DELEUZE1996} La diferencia no es producida por el entendimiento, sino que surge de la misma experiencia: toda percepción discernible es una existencia separada. Esto rompe con la idea de representación como un sujeto que ordena el caos del mundo objetivo.

Si en el estudio del empirismo se puede ubicar el primer antecedente, la referencia a Friedrich Nietzsche en Deleuze es central por dos motivos: primero, porque será leyendo la noción de \enquote{voluntad de poder} que formula su noción de diferencia y, segundo, porque sirve para comprender de qué modo se trata de pensar a Nietzsche contra Hegel, donde aparece claramente un primer esbozo de la relación entre lo afirmativo y lo negativo. En este sentido, el uso de Nietzsche por autores como Foucault, Deleuze o Derrida estará signado por el abandono de la dialéctica.\footcite{@6971-SAZBON2009,@6972-CASTRO2002} En la lectura de Nietzsche la diferencia aparece en relación a la noción de fuerza, pues todo objeto se configura desde una pluralidad de fuerzas que actúan unas sobre otras. No será sino esa multiplicidad como relación diferencial de fuerzas lo que aparezca en el concepto de diferencia. Para Deleuze, desde el momento en que la fuerza está relacionada con otra fuerza se llama voluntad, por lo que la voluntad es el elemento diferencial de la fuerza: \enquote{Que cualquier fuerza se relaciona con otra, sea para obedecer sea para mandar, he aquí lo que nos encamina hacia el origen: el origen es la diferencia en el origen, la diferencia en el origen es \emph{jerarquía}, es decir la relación de una fuerza dominante con una fuerza dominada, de una voluntad obedecida con una voluntad obediente}.\footcites[16]{@6973-DELEUZE1998}[Indudablemente si en la noción de voluntad de poder se encuentra la genealogía de la diferencia deleuzeana, la repetición surge del eterno retorno: \enquote{(\dots) el eterno retorno se dice solamente del devenir, de lo múltiple. Es la ley de un mundo sin ser, sin unidad, sin identidad. Lejos de \emph{presuponer} lo Uno o lo Mismo, constituye la unidad exclusiva de lo múltiple en cuanto múltiple, la única identidad de lo que difiere: el volver es el único \enquote{ser} del devenir}.][163]{@6974-DELEUZE2005}

La lectura de Nietzsche nos permite ingresar en el núcleo de la diferencia tal como será elaborado en \emph{Diferencia y repetición}, libro que puede ser caracterizado como un tratado sobre la diferencia. El pensamiento de la diferencia en Deleuze se caracteriza por su distancia respecto a la negatividad hegeliana, su ruptura con el primado de la identidad y la crítica a la representación: \enquote{Pues la diferencia no implica lo negativo, y no admite ser llevada hasta la contradicción más que en la medida en que se continúe subordinándola a lo idéntico. El primado de la identidad, cualquiera sea la forma en que esta sea concebida, define el mundo de la representación}.\footcite[15]{@6961-DELEUZE2002} Todo el esfuerzo deleuzeano se encuentra en pensar una diferencia sin negación: una \emph{diferencia pura}. Esto implica que existen dos formas de comprender la diferencia, una dialéctica y una pura:

\begin{quote}
	La dialéctica hegeliana consiste en la reflexión sobre la diferencia, pero invierte la imagen. Sustituye la afirmación de la diferencia como tal por la negación de lo que difiere; la afirmación de sí mismo, por la negación del otro; la afirmación de la afirmación, por la famosa negación de la negación. (\dots) Al ocupar la oposición el lugar de la diferencia, se produce el triunfo de las fuerzas reactivas que hallan en la voluntad de la nada el principio que les corresponde.\footcite[272]{@6973-DELEUZE1998}
\end{quote}

De modo que el primer elemento que es necesario retomar de la diferencia en Deleuze es su crítica a Hegel. La crítica a lo negativo no debe entenderse como su simple abandono. Por el contrario se trata en Deleuze de pensar de modo afirmativo la diferencia y desde allí mostrar que la afirmación es primera respecto de la negación. En todo caso, la negación deja de ser una cualidad primera y un poder autónomo, y por ello se subordina a la afirmación. Si la negación se opone a la afirmación, la afirmación sólo difiere de la negación, pues si se señala que la afirmación se opone a la negación ya se introduce la negación o contradicción en su seno. Si la oposición es la esencia de la negación, la diferencia es la esencia de la afirmación. Afirmación que no es la afirmación de lo dado, la aceptación, imposibilidad de decir que no, sino afirmación como creación. Afirmación que, en última instancia, se duplica: es afirmación de la afirmación. Lo negativo resulta de la afirmación de la diferencia: \enquote{Lo negativo no está presente en la esencia como aquello de donde la fuerza extrae su actividad: al contrario, resulta de esta actividad, de la existencia de una fuerza activa y de la afirmación de su diferencia. Lo negativo es un producto de la propia existencia: la agresividad necesariamente asociada a una existencia activa, la agresividad de una afirmación}.\footcite[17]{@6973-DELEUZE1998}

En la distancia con la dialéctica hay que pensar la distancia entre diferencia y oposición, pues la diferencia hegeliana se piensa como oposición. Si en Hegel la fuerza tiene un lugar central, piensa la misma como oposición o contradicción. Esto para Deleuze supone una abstracción de las fuerzas donde se pierde el elemento real del que proceden las fuerzas, esto es, la dialéctica se queda en la abstracción de las relaciones. Por el contrario en la diferencia se piensa la infinita complejidad de la génesis de las fuerzas en relación. Por ello, Deleuze no niega simplemente la negación, sino que muestra su trasfondo diferencial. Para que sea posible una oposición que limita, es preciso un trasfondo de fuerzas diferenciales, una multiplicidad informal y potencial. Es sobre este trasfondo que se trazan las limitaciones y oposiciones. De modo que lo negativo es una imagen invertida de la diferencia: \enquote{No es la diferencia lo que supone la oposición, sino la oposición lo que supone la diferencia, y lejos de resolverla, es decir, de conducirla hasta un fundamento, la oposición traiciona y desnaturaliza la diferencia}.\footcite[94]{@6961-DELEUZE2002}La diferencia no se deja llevar hasta la contradicción porque es más profunda que ella, es la que posibilita incluso la contradicción. Para que exista contradicción u oposición es necesario que la diferencia sea mediada por el concepto y se haya constituido en algo idéntico. Sólo cuando la diferencia es reducida por la identidad a un mismo plano surge la contradicción. \ref{tab: Cuadro 1.2} % referencia únicamente para práctica

En este marco, resulta central la crítica a la representación como la dominación de la diferencia en tanto la subordina al plano del concepto. La identidad del concepto fija las diferencias de modo externo. La representación es la relación del concepto con su objeto, por lo que la pluralidad del mundo es reducida al concepto que la representa y garantiza su unidad. La diferencia es de este modo regulada por el concepto, subsumida en la unidad que representa. Frente a ello, Deleuze propone pensar la diferencia sin concepto: los existentes que se resisten a la correspondencia con el concepto. La diferencia es pensada como la relación entre lo diferente y lo diferente, por fuera de las formas de la representación que la subordinan a lo mismo. En el caso de la representación, la diferencia es externa porque está dada en la representación de un concepto, la \emph{diferencia interna} es no conceptual y no puede ser representada por un observador externo:

\begin{quote}
	El error de la filosofía de la diferencia, de Aristóteles a Hegel, pasando por Leibniz, fue tal vez haber confundido el concepto de la diferencia con una diferencia simplemente conceptual, contentándose con inscribir la diferencia en el concepto en general. En realidad, mientras se inscriba la diferencia en el concepto en general, no tendremos ninguna Idea singular de la diferencia, permaneceremos tan sólo en el elemento de una diferencia ya mediatizada por la representación.\footcites[58]{@6961-DELEUZE2002}[La noción de \enquote{diferencia interna} resulta central puesto que se permite comprender la distancia con una noción de diferencia externa que no sólo parte de la imposición del concepto exterior, sino de la oposición. En un texto donde rastrea la diferencia en Bergson, Deleuze escribe: \enquote{La filosofía mantiene una relación positiva y directa con las cosas solamente en la medida en que pretende captar la cosa misma de lo que ella es, en su diferencia con respecto a todo lo demás, es decir, en su \emph{diferencia interna}. (\dots) Si existen diferencias de naturaleza entre individuos de un mismo género, habremos de reconocer que, efectivamente, la diferencia no es simplemente espacio-temporal, ni tampoco genérica o específica y, en suma, que no es exterior ni superior a la cosa misma}.][46]{@6975-DELEUZE2005}
\end{quote}

Existen dos formas de la representación que niegan la diferencia: por un lado, la representación finita, propia del aristotelismo, representa la diferencia mediatizándola al subordinarla a los géneros; por otro lado, la representación infinita, propia del hegelianismo, constituye a la mediación como fundamento en la diferencia entre la totalidad y el sujeto. En ambos casos la diferencia se subordina a la identidad pues se inscribe en el concepto general. El problema de la representación es que la mediatización del concepto termina por eliminar la diferencia, es decir, reducirla a la identidad del concepto. Esto surge como reaseguro ante una realidad múltiple, ante la coexistencia de una pluralidad de fuerzas que imposibilitan esa misma representación. Frente a ello no se trata de pluralizar los puntos de vista, multiplicar las representaciones, sino destituir la representación:

\begin{quote}
	Es necesario que la diferencia se convierta en el elemento, en la unidad última que remita, pues, a otras diferencias que no la identifiquen sino que la diferencien. Es preciso que cada término de una serie, siendo ya diferencia, sea puesto en una relación variable con otros términos, y constituya así otras series desprovistas de centro y de convergencia. Hay que afirmar, en la serie misma, la divergencia y el descentramiento. Cada cosa, cada ser, debe ver su propia identidad sumida en la diferencia, ya que cada uno no es más que una diferencia entre diferencias. Hay que mostrar la diferencia \emph{difiriendo}.\footcite[101]{@6961-DELEUZE2002}
\end{quote}

La pregunta es entonces cómo pensar una diferencia no conceptual, es decir, no mediada por la representación. La filosofía de la diferencia deleuzeana niega entonces que toda determinación sea negación, es decir, se trata de pensar una \emph{determinación sin negación}. Por lo que hay que romper el dualismo hegeliano entre lo indeterminado indiferente (la noche de gatos pardos) y la determinación como negación (el trabajo de la dialéctica). Para ello Deleuze señala que la diferencia es determinación, pero no se trata de una determinación extrínseca mediada por el concepto, sino la determinación como distinción unilateral. Por esto mismo es posible afirmar que la diferencia se hace, se trata de \enquote{hacer la diferencia}. Con esto hay que evitar la confusión entre la búsqueda de un concepto de diferencia y la inscripción de la diferencia en el concepto general.

El punto de partida de la ontología deleuzeana es la afirmación de la \emph{univocidad del ser}. Esto no significa que ser se diga en un mismo y único sentido, sino que se dice en un único y mismo sentido de todas las diferencias individuantes. Las modalidades del ser son diferentes pero el ser es el mismo para todas ellas. Deleuze señala que el ser se dice en un mismo sentido de todo lo que se dice, pero aquello de lo que se dice difiere, por esto la univocidad del ser se dice de la diferencia misma:

\begin{quote}
	Que el ser sea unívoco, que sólo pueda decirse de una única y misma manera, es paradójicamente la mayor condición para que la identidad no domine la diferencia, y que la ley de lo Mismo no la fije como simple oposición en el elemento del concepto; el ser puede decirse de la misma manera ya que las diferencias no están reducidas de antemano por las categorías, ya que no se reparten en un diverso siempre reconocible por la percepción, que no se organizan según la jerarquía conceptos de las especies y los géneros.\footcite[42]{@6976-FOUCAULT1995}
\end{quote}

Existen dos modos de pensar la distribución y las jerarquías en las diferencias: en el caso de la representación la distribución se da por propiedades limitadas en la misma representación, en el otro caso la distribución se da sin un principio, ya no hay reparto de una distribución representada, sino la misma partición de quienes se distribuyen en un espacio abierto. Lo mismo sucede con la jerarquía, pues en un caso se mide por la distancia o cercanía respecto de un principio, en el otro se trata de grados de la potencia no en función de un principio sino de lo que esa misma potencia puede. En este sentido existe una distribución nómade del ser unívoco.

La diferencia no puede ser conducida a una instancia previa, la diferencia es ontológica en tanto relaciona lo diferente con lo diferente sin ninguna mediación. La diferenciación de la diferencia como \emph{diferenciante}. El término \enquote{diferenciante} nombra la diferencia de segundo grado, pues si existen dos o más series definidas por las diferencias que las componen, la comunicación de estas series de diferencias también es diferencial. Al entrar en contacto series heterogéneas algo pasa, estallan acontecimientos. Un dinamismo en la resonancia de las series acopladas. Lo que lleva a una pregunta: ¿es la diferencia la que relaciona lo diferente con lo diferente? Para que sea posible el acoplamiento de dos series heterogéneas resultaría necesario un elemento común que las haga entrar en contacto. Para resolver esto, Deleuze introduce la idea de un \emph{precursor sombrío}, siendo aquello que posibilita la comunicación entre series heterogéneas. Claro que el problema es atribuirle identidad o semejanza a ese precursor, o mejor, identidad y semejanza son conceptos de la reflexión para pensar ese precursor: \enquote{Dadas dos series heterogéneas, dos series de diferencias, el precursor actúa como el diferenciante de estas diferencias. De este modo, las relaciona de inmediato en virtud de su propia potencia: es el en-sí de la diferencia o lo \enquote{diferentemente diferente}, es decir, la diferencia en segundo grado, la diferencia consigo que relaciona lo diferente con lo diferente por sí mismo}.\footcite[186]{@6961-DELEUZE2002}

Este precursor es llamado por Deleuze \enquote{dispar}, y no procede de modo anticipatorio o estableciendo una teleología de la vinculación, sino que sólo se vuelve visible de modo retroactivo, sólo en cuanto se pueden analizar las series vinculadas. Por esto mismo no tiene una identidad dada, es una X que falta. Este precursor se desplaza y se disfraza constantemente, la identidad lógica o la semejanza empírica sólo son formas de ocultar el precursor. La necesidad de un tercero que vincule dos cadenas de diferencias no es una condición sino de la representación. El precursor que vincula series heterogéneas constituye un espacio de desplazamiento que determina una magnitud relativa de las diferencias relacionadas. Donde la magnitud no se mide en función de un principio externo a las series vinculadas, sino que surge de la diferenciación interna que produce. Cada serie se explica sólo en tanto está implicada en otras series. Estas series son coexistentes, se dan siempre en la simultaneidad. Esto es lo que se denomina \emph{multiplicidad}: una organización que no tiene la necesidad de la unidad para formar un sistema. La multiplicidad no está subordinada a ningún principio exterior puesto que manifiesta la misma diferencia. Los elementos de la multiplicidad están determinados recíprocamente, por lo que no existe ninguna independencia. Esto implica \enquote{(\dots) plantar la cuestión del ser del devenir en términos de disposición relativa de una pluralidad de elementos que no sólo se constituyen diferencialmente unos en relación a otros sino que, además, no constituyen el circuito de su mutua reciprocidad más que en tanto logran hacer pasar algo que necesariamente los excede y atraviesa por el medio}.\footcite[15]{@6977-GALLEGO2008}

Según las indicaciones establecidas hasta aquí se podría entender la diferencia sólo en el plano de lo dado. Por el contrario Deleuze señala que no se trata de la diferencia entre individuos empíricos, sino de aquel principio trascedente que actúa individuando, es decir, los factores individuantes. Se trata entonces de una \emph{diferencia individuante}. Por lo que el ser se dice de la diferencia, donde nuestra individualidad permanece equívoca en un ser unívoco. Ahora bien, para entender la individuación resulta central atender a la distinción entre lo virtual y lo actual. Esto porque la diferencia no es lo diverso en lo dado, sino aquello por lo cual lo dado es diverso. La diferencia, entonces, es virtual. Esto no significa que sea posible o que carezca de realidad, sino que tiene un estatuto trascendental: la diferencia no es lo dado, sino aquello que configura lo dado. La multiplicidad en tanto diferenciante es la idea, así una relación múltiple ideal como relación diferencial se actualiza en relaciones espacio-temporales: \enquote{(\dots) una multiplicidad interna, es decir, una sistema de relación múltiple no localizable entre elementos diferenciales que se encarna en relaciones reales y términos actuales}.\footcite[278]{@6961-DELEUZE2002} Por ello no hay oposición entre génesis y estructura, la génesis de una multiplicidad es su misma estructura, se trata sólo del paso de lo virtual a su actualización. Una estructura no es sino un sistema de relaciones y elementos diferenciales.

La diferencia se comprende atendiendo al doble sentido del término \enquote{diferenciación}: la diferenciación virtual en el plano de la idea y la diferenciación como actualización de esa virtualidad. Esto es lo que se encuentra en la distinción deleuzeana entre \emph{differentiation} y \emph{differenciation}. El primer sentido hace referencia a la diferenciación estructural, en el plano de la Idea; en el segundo sentido la diferenciación es la actualización de esa virtualidad. El paso de lo virtual a lo actual es una dramatización, es decir, el paso de lo ideal a lo sensible se da por un proceso de diferenciación de intensidades. La diversidad de lo dado es posible por la diferencia que es su razón suficiente: \enquote{Todo lo que pasa y aparece es correlativo de órdenes de diferencias, diferencia de nivel, de temperatura, de presión, de tensión, de potencial, \emph{diferencia de intensidad}}.\footcite[333]{@6961-DELEUZE2002} La intensidad no es un calificativo de la diferencia, en realidad se trata de una tautología, la intensidad es la forma de la diferencia pues toda intensidad es diferencial. La diferencia es lo que crea lo diverso, la diferencia de intensidad es lo profundo de donde surgen la extensión y la cualidad. La intensidad tiene tres caracteres: primero, la intensidad siempre se da como diferencia de cantidad, es decir, lo inigualable en la cantidad misma; segundo, la intensidad hace de la diferencia un objeto de afirmación, y esto implica que no posee ninguna negatividad; tercero, la intensidad es una cantidad implicada que no puede dividirse sin cambiar de naturaleza. La intensidad es entonces lo determinante en el proceso de actualización: es la intensidad lo que se dramatiza.\footcite{@6975-DELEUZE2005}

En resumidas cuentas, es posible indicar que el pensamiento deleuzeano de la diferencia da cuenta de una diferencia que escapa a la representación. Y esto debido a que la representación sólo aborda la diferencia como algo externo subordinado a la identidad del concepto. Es esto lo que le permite criticar la elaboración de la diferencia en el pensamiento clásico y, ante todo, en la dialéctica hegeliana. El desafío resulta pensar una diferencia más allá de la negatividad. Que, como hemos repetido, no significa simplemente eliminar la negación o la oposición, sino mostrar su carácter derivado. Con ello se rompe la oposición hegeliana entre lo indeterminado indiferenciado y la determinación negativa, pues la diferencia es determinación como distinción. De ahí que sea eminentemente activa, la diferencia se hace en tanto se comprende como proceso de diferenciación. Diferenciación estructural y diferenciación como actualización. Por lo que la diferencia no es simplemente empírica, interior al mundo, sino que es trascendental. Esto se comprende desde el juego de palabras con el término diferenciación: la diferencia es ese precursor que hace que lo diferente sea diferente. La diferencia se convierte, de este modo, en irreductiblemente múltiple y vinculada con otras diferencias: \enquote{(\dots) la verdad de la realidad de aquello que es como diferenciado en un sentido triple: 1) porque se diferencia respecto de otros seres, 2) porque resulta diferenciado \emph{en} el ser, 3) porque es, en sí mismo, bien una diferencia, bien la integración de un cierto complejo de diferencias mutuamente ligadas}.\footcite[17]{@6977-GALLEGO2008}

\section{La diferencia negativa: Jacques Derrida}

Si la dificultad central a la hora de exponer la forma en la que Deleuze concibe la diferencia se encuentra en que toda su obra puede ser calificada como filosofía de la diferencia, en el caso de Jacques Derrida la dificultad surge de la diseminación de la noción de diferencia en diversos escritos. Ya en su primer texto, una extensa introducción a \emph{El origen de la geometría} de Edmund Husserl, de 1962, finaliza con una referencia a la diferencia. En Husserl, los objetos matemáticos son ideales en tanto tienen una pureza incontaminada por la facticidad. Pero esa idealidad tiene una génesis, es decir, surge en determinado contexto histórico y en cierta subjetividad. La cuestión es entonces cómo pensar el origen de la geometría en tanto su idealidad parece prescindir de todo contexto fáctico de emergencia. Derrida nota que Husserl para pensar el paso de la subjetividad del inventor a la verdad ideal introduce no sólo la referencia al lenguaje (resulta inherente a la objetividad la intersubjetividad posibilitada por la mediación lingüística) sino a la escritura (como posibilidad de transmisión más allá de un contexto histórico particular). Esto desmantela los postulados husserlianos en tanto la escritura, cuya facticidad es irreductible, tiene un estatuto genético respecto de la idealidad de los objetos matemáticos. Finalizando este texto escribe Derrida: \enquote{Trascendental sería la Diferencia. Trascendental, sería la inquietud pura e interminable del pensamiento dispuesto a \enquote{\emph{reducir}} la Diferencia, superando la infinitud fáctica hacia la infinitud de su sentido y de su valor, es decir, manteniendo la Diferencia}.\footcite[162]{@6978-DERRIDA2000} La diferencia no se ubica en la facticidad, sino que tiene un lugar trascendental, o mejor, la diferencia es cuasi-trascendental puesto que se ubica entre la infinitud fáctica y la infinitud ideal.\ref{tab:Cuadro 1.1}

Esta primera aparición de la diferencia será complejizada a lo largo de los escritos posteriores llevando a Derrida a construir un neografismo para singularizar su perspectiva: la \emph{différance}. La forma de escribir el término introduce una variación gráfica que resulta inaudible. Esto sólo se comprende en el marco de una reflexión constante sobre la escritura en los textos tempranos del autor. Derrida muestra en distintas lecturas la persistencia en la tradición de un esquema que ubica la escritura en un lugar derivado. Este esquema parte del pensamiento como algo inmediato, luego la voz como la mediación más cercana a ese pensamiento y por último la escritura como mediación que sólo \enquote{representa} la voz. Esto significa que una y otra vez se repite la ilusión de un pensamiento sin lenguaje \rdm{sin mediación}, presente a sí mismo. Y el privilegio de la voz resulta de su materialidad difusa, pues al borrarse restituye esa ilusión. La escritura será condenada entonces no sólo porque es una mera representación del habla como primera mediación, sino porque su materialidad viene a mostrar el carácter irreductible de las mediaciones en el mismo seno del pensamiento. Al mismo tiempo, la escritura en tanto mera representación del habla va a ser caracterizada como un significante de otro significante. Paradójicamente, esa caracterización nombra la diferencia que no sólo no es derivada, sino que es lo que constituye el mismo proceso de significación. Si Derrida escribe \emph{différance} es porque existe una íntima unidad entre grafía y diferencia, es decir, se cuestiona la reducción de la diferencia por la presencia o la inmediatez.

La diferencia no puede ser nombrada con un concepto o una palabra, pues da cuenta del movimiento de \enquote{constitución} de conceptos o palabras, de unidades significativas. Por ello Derrida señala que la \emph{différance} es un \enquote{haz} indicando dos cosas: por un lado, que no se busca reconstruir y contextualizar los diferentes usos textuales que se le ha dado al término, sino de referir a un sistema de economía general; por otro lado, la palabra haz refleja la idea de tejido, de un cruce de líneas de sentido: \enquote{(\dots) los elementos del sistema no tienen nada de atómico (en el sentido clásico, al menos), hay que decir que tales elementos no son otra cosa que haces de huellas. Estas huellas no son lo que cierta lingüística denomina rasgos distintivos, sino sólo las huellas de la ausencia del otro \enquote{elemento} que, por otro lado, no está ausente en el sentido de \enquote{presente en otro lado}, sino formado, él también, por huellas}.\footcite[95]{@6979-BENNINGTON1994} La diseminación de la \emph{différance} no se refiere sólo a su inscripción en diversos textos, sino a la imposibilidad de fijarla como elemento simple. Aún más, la diferencia en Derrida viene a señalar la imposibilidad de la simplicidad de los elementos en cuanto para su misma constitución necesitan la referencia a otros elementos.

La noción de \emph{differánce}, señala Derrida, tiene tres antecedentes: Nietzsche, Freud y Heidegger. En primer lugar, en el caso de Nietzsche, Derrida señala que ha puesto en tela de juicio la idea de una conciencia presente a sí misma, pues la gran fuente de actividad es el inconsciente. La consciencia es un mero efecto de esas fuerzas que le son extrañas. Ahora bien, Derrida siguiendo la lectura deleuzeana de Nietzsche, señala que esa fuerza no estaría nunca presente puesto que es un juego del diferir de fuerzas. El pensamiento de Nietzsche sirve aquí para mostrar que la \emph{différance} derridiana implica una lucha de fuerzas, es siempre una diferencia entre fuerzas:

\begin{quote}
	Todo el pensamiento de Nietzsche ¿no es una crítica de la filosofía como indiferencia activa ante la diferencia, como sistema de reducción o de represión a-diaforística? Lo cual no excluye que según la misma lógica, según la lógica misma, la filosofía viva \emph{en}y\emph{de}la \emph{différance}, cegándose así a lo \emph{mismo} que no es lo idéntico. Lo mismo es precisamente la \emph{différance} (con una a) como paso desviado y equívoco de un diferente a otro, de un término de la oposición a otro. Podríamos así volver a tomar todas las parejas en oposición sobre las que se ha construido la filosofía y de las que vive nuestro discurso para ver ahí no borrarse la oposición, sino anunciarse una necesidad tal que uno de los términos aparezca como la \emph{différance} del otro, como el otro diferido en la economía del mismo.\footcite[18]{@6980-DERRIDA1989}
\end{quote}

En segundo lugar, resulta central la referencia a Freud en tanto anticipa los dos sentidos de la \emph{différance}: el diferir como distinción, desviación, espaciamiento; y el diferir como demora, reserva, temporalización. No es posible hablar del origen de la memoria y el psiquismo sin apelar a la diferencia. El inconsciente introduce una alteridad que no puede ser reapropiada por la presencia, el inconsciente es un pasado que nunca fue presente y que no se puede volver presente. Una alteridad que se encuentra en la noción de huella y que permite comprender en qué sentido el planteo de Freud es importante aquí.

\begin{quote}
	Una cierta alteridad \rdm{Freud le da el nombre metafísico de inconsciente} es definitivamente sustraída a todo proceso de presentación por el cual lo llamaríamos a mostrarse en persona. En este contexto y bajo este nombre el inconsciente no es, como es sabido, una presencia para sí escondida, virtual, potencial. Se difiere, esto quiere decir sin duda que se teje de diferencias y también que envía, que delega representantes, mandatarios; pero no hay ninguna posibilidad de que el que manda \enquote{exista}, esté presente, sea el mismo en algún sitio y todavía menos de que se haga consciente.\footcite[21]{@6980-DERRIDA1989}
\end{quote}

En tercer lugar, Derrida señala que todo el horizonte epocal en el que se ubica es la interrogación del ser como presencia realizada por Heidegger. Esto conduce a la relación que se establece entre el término derridiano y la diferencia óntico-ontológica heideggeriana: \enquote{Por una cierta cara de sí misma, la \emph{différance} no es ciertamente más que el \emph{despliegue} histórico y de época del ser o de la diferencia ontológica. La \emph{a} de la \emph{différance} señala el \emph{movimiento}de este despliegue}.\footcites[23]{@6980-DERRIDA1989}[La diferencia ontológica sería, así, derivada: \enquote{(\dots) ente y ser, óntico y ontológico, \enquote{óntico-ontológico} serían, en un estilo original, \emph{derivados} respecto de la diferencia; y en relación con lo que más adelante denominaremos la \emph{différance}, concepto económico que designa la producción del diferir, en el doble sentido de esta palabra. La diferencia óntico-ontológica y su fundamento (\emph{Grund}) en la \enquote{trascendencia del Dasein} [\emph{Vom Wesen des Grundes}, p. 16] no serían absolutamente originarios. La \emph{différance} sería más \enquote{originaria}, pero no podría denominársela ya \enquote{origen} ni \enquote{fundamento}, puesto que estas nociones pertenecen esencialmente a la historia de la onto-teología, es decir al sistema que funciona como borradura de la diferencia}.][32]{@6992-DERRIDA1998} Este despliegue es también un intento de dar un paso más allá de Heidegger. La \emph{différance}, por un lado, y a diferencia de Heidegger, será un término que pertenece a la metafísica en tanto nada se puede nombrar si no es en sus límites; pero, por otro lado, intenta escapar a esa determinación histórica. La singularidad de la posición de Derrida se encuentra en el cruce de ambas posibilidades, aquello que ha de llamar a lo largo de sus primeros escritos \enquote{clausura}. Con este término se indica que no se trata del fin o el cierre de la metafísica, lo cual sería imposible, sino de una determinada forma de trabajar al interior de la tradición. Pretender abandonar definitivamente la tradición metafísica resulta imposible puesto que inevitablemente hablamos un lenguaje atravesado por la misma. Por ello, se trata de habitar en su interior para dislocar los significados constituidos. Es por esto que Derrida retoma el discurso heideggeriano, pero sin la creencia en una etapa presocrática donde sea posible nombrar el ser. Se trata de romper con la \enquote{nostalgia} de un nombre propio inscripta en el pensador alemán. No existe nombre para la \emph{différance}, no existe una unidad nominal, por ello se disloca todo el tiempo en una cadena de sustituciones. Y esto implica una fuerte afirmación, un gesto nietzscheano de risa y danza. Lo que derrumba la otra cara de la nostalgia que es la esperanza, dirigida a encontrar el nombre único, el nombre-Dios.

Las referencias a Nietzsche, Freud y Heidegger, le sirven a Derrida, como a Deleuze, para marcar un claro distanciamiento de la dialéctica. Pero se trata de dos distanciamientos diferentes, y donde se juega posiblemente la distancia entre sus nociones de diferencia. En el caso de Derrida, no se trata de abandonar la dialéctica hegeliana, sino de llevarla a sus límites. De hecho la relación con Hegel encuentra su lugar privilegiado en los primeros escritos de Derrida en la lectura de Georges Bataille. No es posible simplemente evitar la lectura de Hegel, lo cual supondría en todo caso quedar atrapados en las redes de su pensamiento, sino de esbozar un \enquote{hegelianismo sin reserva}. Para Derrida la dialéctica se puede comprender como una economía del sentido, es decir, como un pensamiento cuya circularidad se apropia de la negatividad en vistas a la construcción de un sentido. Todo aquello que parece exceder la razón termina siendo reapropiado. Frente a ello, Bataille rompe con el círculo económico que siempre implica un reencuentro final, es una ruptura con la diferencia hegeliana donde el movimiento finaliza en el reencuentro:

\begin{quote}
	Pues el carácter económico de la \emph{différance} no implica de ninguna manera que la presencia diferida pueda ser todavía reencontrada, que no haya así más que una inversión que retarda provisionalmente y sin pérdida la presentación de la presencia, la percepción del beneficio o el beneficio de la percepción. Contrariamente a la interpretación metafísica, dialéctica, \enquote{hegeliana} del movimiento económico de la \emph{différance}, hay que admitir aquí un juego donde quien pierde gana y donde se gana y pierde cada vez. Si la presentación desviada sigue siendo definitiva e implacablemente rechazada, no es sino un cierto presente lo que permanece escondido o ausente; pero la \emph{différance} nos mantiene en relación con aquello de lo que ignoramos necesariamente que excede la alternativa de la presencia y de la ausencia.\footcites[21]{@6980-DERRIDA1989}[La \emph{différance} al mismo tiempo que no puede entenderse como un antihegelianismo indicativo, no se comprende desde la circularidad de sentido de la dialéctica: \enquote{La \emph{différance} debe señalar el punto de ruptura con el sistema de la \emph{Aufhebung} y de la dialéctica especulativa. Esta conflictualidad de la différance, que no se puede llamar contradicción, que a condición de demarcarla por un largo trabajo de la de Hegel, no dejándose nunca relevar totalmente, marca sus efectos en lo que he llamado el texto en general, en un texto que no se contiene en el reducto del libro o de la biblioteca y no se deja nunca dirigir por un referente en el sentido clásico, por una cosa o por un significado transcendental que regularía todo el movimiento. No es, como se puede ver, por prurito de apaciguamiento o reconciliación por lo que recurro de buena gana a la marca \enquote{\emph{différance}} antes que al sistema de la diferencia-y-de-la-contradicción}.][58]{@6993-DERRIDA1877}[Al respecto, señala Caterina Resta: \enquote{Se trata de pensar una mediación sin oposición: «La \emph{différance} afirma una \enquote{diferencia radical}, una radical alteridad y heterogeneidad que todavía dan lugar a una relación, a una \emph{mediación}. La di-ferencia es quizá esto: relación con lo inapropiable, con lo sin-relación}.][123]{@6981-RESTA1990}
\end{quote}
% La tercera cita no se resalta en TeXstudio por limitación del resaltado de sintaxis, pero compila bien.

Desde estos antecedentes, tres señalamientos es posible establecer para pensar la \emph{différance}. Primero, la diferencia se refiere a la diferencialidad de las diferencias, la fuerza que conserva agrupado al sistema ante su dispersión, es decir, su mantenimiento. La \emph{différance} se entiende desde la diferencia y la arbitrariedad como estructura general de la lengua propuesta por Saussure. Como indica este autor, la diferencia implica que no existen términos positivos, esto es, que no hay concepto o significado presente a sí mismo, sino que se constituye en la cadena de intercambios. Existen dos rasgos de esta diferencia que la hacen estar y escapar del sistema de la lengua: las diferencias actúan, constituyen los términos, pero esas diferencias son efectos de la misma lengua, son históricas. Segundo, la diferencia se refiere a la demora o el retraso que hace que el sentido siempre se anticipe o se restablezca posteriormente. En este caso es la diferencia como temporalización, como diferir. Verbo que significa dejar para más tarde, temporizar, recurrir a una mediación temporal que suspende el cumplimiento o satisfacción de la voluntad. \footcite[Por lo que se juega aquí también una comprensión de la temporalidad: \enquote{Estos conceptos de \emph{différance}} y de retardo originarios son impensables bajo la autoridad de la lógica de la identidad o incluso bajo el concepto de tiempo. El absurdo mismo que se señala así \emph{en los términos} permite pensar, con tal que esté organizado de una cierta manera, el más allá de esta lógica y de este concepto. Bajo la palabra \emph{retardo}, hay que pensar otra cosa que una relación entre dos \enquote{presentes}; hay que evitar la representación siguiente: sólo ocurre en un presente B lo que debía (habría debido) producirse en un presente A (\enquote{anterior}).][280]{@6982-DERRIDA1989}  Tercero, la posibilidad de toda distinción conceptual, por ejemplo, entre sensible e inteligible, esto es el espaciamiento. Desde los tres sentidos es posible indicar que la diferencia nombra la diferencialidad de todo proceso de significación como temporalización y espaciamiento. La \emph{différance} designa, así, el movimiento por el cual todo sistema de repeticiones se constituye históricamente como entramado de diferencias. Esto implica que cada cosa presente se constituye reteniendo la marca del elemento pasado y dejándose marcar por el elemento futuro. La \emph{différance} es ese espaciamiento y temporalización como producción de diferencias:

\begin{quote}
	Lo que se escribe como \enquote{\emph{différance}} será así el movimiento de juego que \enquote{produce}, por lo que no es simplemente una actividad, estas diferencias, estos efectos de diferencia. Esto no quiere decir que la \emph{différance} que produce las diferencias esté antes que ellas en un presente simple y en sí mismo inmodificado, in-diferente. La \emph{différance} es el \enquote{origen} no-pleno, no-simple, el origen estructurado y diferente (de diferir) de las diferencias.\footcite[12]{@6980-DERRIDA1989}
\end{quote}

Ahora bien, la diferencia en Derrida no sólo indica la diferencialidad en tanto distinción y retraso, sino que la diferencia también es \emph{polemos}. En este caso, es la diferencia como diferendo irreductible, como lucha de fuerzas. Aún más, lo que no ha podido ser pensado es esta conjunción en la diferencia de diferir y diferendo: \enquote{(\dots) la palabra diferencia (con \emph{e}) nunca ha podido remitir así a diferir como temporización ni al desacuerdo como \emph{polemos}. Es esta pérdida de sentido lo que debería compensar \rdm{económicamente} la palabra \emph{différance} (con \emph{a})}.\footcite[8]{@6980-DERRIDA1989} En esta notación de Derrida se encuentra el elemento central que ha sido referido: la \emph{différance} no es sólo una reinscripción de la diferencia saussuriana a partir del diferir temporal, sino que se relaciona con el \emph{polemos,} es decir, con un desacuerdo que no remite simplemente al conflicto fáctico entre dos posiciones. Se trata del desacuerdo como lucha de fuerzas, o economía de la violencia, ubicado en la misma diferencialidad de todo proceso de significación.

Es importante notar que las diferencias no son producidas por un sujeto primigenio, o por una conciencia \emph{a priori} que tuviera tal facultad de producción. El sujeto mismo es un efecto de la lengua, de ese diferir. El privilegio de la conciencia remite al privilegio otorgado al presente. La \emph{différance} solicita tal privilegio, interroga la presencia cuando se piensa como punto de partida absoluto. Si en el caso de Deleuze lo negativo resultaba derivado respecto de la diferencia, en Derrida la presencia es una determinación o un efecto de la diferencia:

\begin{quote}
	Este privilegio es el éter de la metafísica, el elemento de nuestro pensamiento en tanto que es tomado en la lengua de la metafísica. No se puede delimitar un tal cierre más que solicitando hoy este valor de presencia del que Heidegger ha mostrado que es la determinación ontoteológica del ser; y al solicitar así este valor de presencia, por una puesta en tela de juicio cuyo \emph{status} debe ser completamente singular, interrogamos el privilegio absoluto de esta forma o de esta época de la presencia en general que es la consciencia como querer-decir en la presencia para sí.\footcite[17]{@6980-DERRIDA1989}
\end{quote}

La \emph{différance} es un movimiento de diferenciación, pero también de diferir. Y diferir no es sólo el retraso originario, sino \emph{polemos}¸ lucha. Se trata de una relación de fuerzas como inscripción de la alteridad, pero nunca una inscripción equivalente. Se inscribe la alteridad en una diferencia de fuerzas:

\emph{Différance} designa también, en el mismo campo problemático, a esa economía \rdm{de guerra} que pone con relación a la alteridad radical o a la exterioridad absoluta de lo exterior con el campo cerrado, agonístico y jerarquizante de las oposiciones filosóficas, de los \enquote{diferentes} o de la \enquote{diferencia}. Movimiento económico de la huella que implica a la vez su señal y su desaparición \rdm{el margen de su imposibilidad} según una relación que ninguna dialéctica especulativa del mismo y del otro podría denominar por lo mismo que es una operación de dominio.\footcite[9]{@6983-DERRIDA1997}

La noción de diferencia en Derrida se constituye desde un entramado de sentidos irreductible. Irreductible porque al buscar una pureza del concepto de diferencia se reconduce la diferencia a la no-diferencia, es decir, a la presencia plena. En Derrida, entonces, la diferencia tiene un estatuto cuasi-trascendental. Ese \enquote{cuasi} nombra la facticidad o equivocidad que es imposible reducir en lo trascendental (no existe una idealidad trascendental pura). Una diferencia que remite a un movimiento de diferenciación como el lugar donde se producen las diferencias, así por ejemplo las oposiciones, y esto porque es \enquote{lo mismo} en el que las posiciones se anuncian. En este sentido, no se trata de diferencias fácticas, sino de la producción de las diferencias: la diacriticidad como condición de toda significación. Esta producción de diferencias es, primero, diferir como remisión o retraso; segundo, es distinción o separación; tercero, es diferendo como lucha de fuerzas:

\begin{quote}
	La actividad o la productividad connotadas por la \emph{a} de la \emph{différance} remiten al movimiento generativo en el juego de las diferencias. Estas diferencias no caen del cielo y no se inscriben de una vez por todas en un sistema cerrado,en una estructura estática que una operación sincrónica y taxonómica podría agotar. Las diferencias son los efectos de transformaciones y desde este punto de vista el tema de la \emph{différance} es incompatible con el motivo estático, sincrónico, taxonómico, ahistórico, etc. del concepto de \emph{estructura}.\footcite[37]{@6983-DERRIDA1997}
\end{quote}

\begin{table}[!ht]
	\sf\footnotesize\setlength\tabcolsep{4pt}
	\centering
	\begin{tabular}{l | >{\raggedright\arraybackslash}m{3.8cm} | >{\raggedright\arraybackslash}m{4cm}}
		\toprule
		\textbf{Autor} & \textbf{Aporte a la \emph{différance}} & \textbf{Conceptos clave} \\
		\midrule
		Nietzsche &
		Crítica a la conciencia como presencia inmediata. La \emph{différance} como lucha de fuerzas y desplazamiento de oposiciones. &
		Voluntad de poder, inconsciente activo, fuerza, oposición. \\
		\midrule
		Freud &
		El inconsciente como alteridad no reapropiable. Dos sentidos de la \emph{différance}: distinción (espacio) y demora (tiempo). &
		Inconsciente, huella, memoria, alteridad. \\
		\midrule
		Heidegger &
		La \emph{différance} como despliegue del ser más allá de la diferencia ontológica. Derrida retoma y disloca su lenguaje. &
		Diferencia ontológica, ser, clausura, presencia. \\
		\bottomrule
	\end{tabular}
	\caption{Antecedentes filosóficos de la \emph{différance} en Derrida}
	\label{tab: Cuadro 1.2}
\end{table}

\section{Ontologías políticas de la diferencia}

Si la noción de diferencia, tal como hemos señalado, es a la vez índice y factor de una época, resta la pregunta por las aperturas teóricas inauguradas allí en su vinculación con la política. Quisiéramos señalar, ante todo, que a pesar de sus diferencias, en Deleuze y Derrida la política adquiere un estatuto ontológico. En este sentido, el pensamiento de estos autores resulta central para pensar la política fuera de un esquema que la ubica en un lugar derivado como área de la realidad o como área de conocimiento. Contra esta subordinación han reaccionado quienes defienden la \enquote{autonomía de lo político}, pero incluso tal autonomía supone un área de la realidad que puede ser calificada de política que tendría su propia legalidad. Por lo que no es en la defensa de la autonomía de lo político que se comprende un pensamiento de la diferencia, pues la diferencia tal como fue señalado cuestiona la misma posibilidad de autonomía ontológica u epistemológica (en otros términos, cuestionan el \enquote{auto} que compone la palabra puesto que supone una entidad definida en un sentido propio no diferencial). No se trata de postular la diferencia específica de la política respecto a otras áreas, sino que la misma diferencia es un movimiento político.

La primera anotación que es posible realizar, en vistas a clarificar la politicidad de la diferencia, es que en ambos autores la diferencia deja de comprenderse como diferencia relacional. \footcite[En este sentido, la diferencia es también la relación con lo sin-relación, con el afuera: \enquote{Lo otro antes de ser mi otro, es decir, el opuesto o la negación de mí \emph{mismo}, es el \emph{afuera} que no deja de habitarme, de repetirme y alterarme al punto en que yo no tengo ya identidad alguna, de espaciarme, diferenciarme, divergirme, aun cuando en este espaciamiento yo no haya tenido nunca identidad ni egoidad alguna, no haya tenido jamás palabra}.][126]{@6985-MENGUE2008} Pensar la diferencia relacional supone ubicarla en una instancia secundaria respecto de elementos previamente constituidos, por lo que la misma constitución de esos elementos no entra en discusión, sino solamente su modo de vinculación. Este cuestionamiento a un elemento definido \emph{a priori} también es realizado por la dialéctica hegeliana. Desde una perspectiva dialéctica la identidad de un elemento surge desde las mediaciones que atraviesa, esto es, la identidad es un resultado de un proceso de negaciones que encuentra al final del camino una totalidad. Por ello mismo el pensamiento de la diferencia surge en discusión con los planteos hegelianos. Si se comparte que lo dado es fruto de mediaciones, la cuestión es cómo pensar las mismas. La crítica a Hegel se ubica en la reducción de la diferencia al enmarcarla en una lógica donde el todo finalmente encuentra una reconciliación consigo mismo. En otros términos, si la identidad o la presencia no son puntos de partida, sí son puntos de llegada en una lógica de la totalidad. Frente a ello se trata de pensar la diferencia sin reconciliación final.

Deleuze y Derrida comparten la disyunción de lo idéntico y lo uno, y así piensan la diferencia misma: \enquote{Ellos compartieron el tiempo filosófico de la diferencia. El tiempo del pensamiento de la diferencia. El tiempo del pensamiento diferente de la diferencia. El tiempo de un pensamiento que debía diferir de aquellos que lo había precedido. El tiempo de un estremecimiento de la identidad: el tiempo, el momento, de un reparto}.\footcite[250]{@6986-NANCY2008} Al mismo tiempo ambos se apartan de una consideración representativa del pensamiento, es decir, de tomar la diferencia como objeto que piensa un sujeto. Esta comunidad no debe interpretarse como un mismo punto de partida, pues en tal caso caeríamos en el absurdo de partir de una identidad, un punto, que a posteriori se diferencia en dos interpretaciones. Por el contrario, los pensamientos de Derrida y Deleuze se dan en la diferencia. Por esto mismo, tampoco se trata de buscar un tercer aspecto que pueda reunir las diferencias, pues no existe una regla común de esa reunión. No hay comunidad, aun cuando ambos abordan la diferencia misma, en sí misma.

La diferencia, entonces, tiene un estatuto ontológico porque no es a posteriori sino que se ubica en la misma configuración de lo dado. De ahí que ni en Deleuze ni en Derrida la diferencia esté ubicada en el orden de las distinciones en lo fáctico o empírico. El estatuto virtual o cuasi-trascendental de la diferencia da cuenta de que lo dado es siempre un efecto, un producto, de un proceso de diferenciación. Claro que esto supone dislocar la misma temporalidad, pues no se puede pensar la diferencia como una instancia previa que causa lo dado, en tal caso la diferencia sería reducida a un elemento simple no diferencial. Tal como muestran ambos autores un pensamiento diferencial rompe con una lógica del origen puro o simple.

Al señalar que lo dado surge de un proceso de diferenciación, proceso por definición abierto (es decir, no regulado en su desarrollo por una determinada lógica), lo dado no es sino un resultado de ciertas diferenciaciones. Y si, como señalan ambos autores, la diferencia es siempre diferencia de fuerzas \rdm{\emph{polemos}}, el mismo proceso de configuración de lo dado es político. Esto es, la política no se ubica en el plano de los vínculos o relaciones entre elementos dados, sino que la política es ontológica porque es el proceso de configuración de lo dado. La política deja de ser un área subordinada que se ocupa de cierta región óntica, tal como las instituciones o los movimientos políticos, para indicar que lo dado es siempre diferencial. Esto no debe ser confundido con una perspectiva que parta de la construcción social de la realidad que en todo caso supone cierto \emph{socius} como sujeto que construye la realidad, y así reproduce en última instancia un esquema moderno donde el sujeto le da sentido al mundo que se representa. Que la diferencia sea ontológica significa que no existe ese antecedente, es decir, se disloca una posición de un sujeto que produce un mundo (donde se entrecruzan la relación causa-efecto y la representación que vincula sujeto y objeto).

Para clarificar el sentido ontológico de la diferencia parece oportuno clarificar su estructura para evitar posibles absurdos. Hemos podido señalar que la diferencia no es secundaria, sino que se ubica es aquello que configura lo dado. ¿Cómo comprender aquí, entonces, el sentido \enquote{constituyente} de la diferencia? Nuestra apuesta se encuentra en pensar la diferencia desde una estructura tripartida. Primero, está la diferencia óntica\index[concepto]{Diferencia!óntica}, las distinciones en el plano de lo dado, diferencias fácticas o empíricas Segundo, está la diferencia ontológica\index[concepto]{Diferencia!ontológica}, aquella que permite diferenciar ente y ser, partiendo de que ente y ser no pueden ser realidades diferentes, sino que son lo mismo. Este señalamiento, central a nuestro juicio, cuestiona aquellos planteos que hacen de la diferencia ontológica el lugar de fundamentos contingentes, repitiendo una posición que termina por reducir el ser a un ente supremo (no cambia demasiado la lógica que a ese ser le llamemos Dios o \enquote{lo político}). En este plano, como supo señalar Heidegger, es imposible hablar de una mediación entre ser y ente, puesto que esto supondría una relación de exterioridad. Por esto, tercero, está la diferencia trascendental, es decir, la serie de mediaciones que hacen de lo dado algo dado. Si no nos queremos atener a lo empírico, a una mera aceptación de lo existente, y tanto Deleuze como Derrida se oponen a esto, lo dado no es sino fruto de un proceso de diferenciación. \footcite[En este sentido acentuamos el carácter trascendental del empirismo de Deleuze. Al respecto señala Agamben: \enquote{(\dots) con Deleuze lo trascendental se aparta decididamente de toda idea de conciencia para presentarse como una experiencia sin conciencia ni sujeto: un empirismo trascendental}][]{@6987-AGAMBEN2008} Proceso de diferenciación que no se ubica en el plano de lo existente, como si se pudiera decir que tal elemento surge de sus relaciones con tal otro, sino que son las \enquote{condiciones de existencia} de lo dado, que son siempre condiciones de posibilidad e imposibilidad al mismo tiempo. \footcite[Por esto mismo nuestra lectura discute las tesis de Gianni Vattimo expuestas en \emph{Las aventuras de la diferencia}. De ningún modo creemos, como señala el autor, que Deleuze y Derrida, retrocedan respecto de Heidegger al eliminar su carácter historial.][]{@6988-VATTINO1999}

Esta estructura tripartita, así, nos permite discutir con quienes ubican la política en una diferencia óntica y con quienes ubican la política en la diferencia ontológica. La política sólo se puede encontrar, desde nuestra perspectiva, en la diferencia trascendental. Como supo señalar Deleuze la diferencia es la diferencia individuante, o mejor, el proceso de diferenciación que hace de un individuo un existente. Lo trascendental se enfrenta a lo inmanente y a lo trascendente, entendidos aquí como algo lo dado y algo exterior. Lo trascendental no es nada diferente a lo dado mismo, sino su condición de existencia que sólo puede ser pensada en una temporalidad retroactiva. Diferencia trascendental, siempre atendiendo a las observaciones críticas sobre lo trascendental hechas por Derrida, que quizá comprenda los procesos polémicos de espaciamiento y temporalización como condición de lo dado. Como señalamos, al reivindicar Deleuze y Derrida las nociones de fuerza y de lucha, destacan que la diferencia en tanto diferenciación conlleva una lucha de fuerzas o una economía de guerra irreductible. De ahí que se pueda indicar que la diferenciación es política, no como un área pasible de ser distinguida de la cultura, la sociedad o la economía, sino como la condición genética \rdm{polémica} de lo existente. Por esto mismo, lo cual es más difícil de pensar, no se trata de tres nociones de diferencia, sino de una complejidad inherente a la misma diferencia.

En la oración fúnebre dedicada a Deleuze, Derrida va a acentuar esta convergencia en un pensamiento diferencial:

\begin{quote}
	Por lo que respecta, aunque esta palabra no es apropiada, a las \enquote{tesis}, y concretamente a aquella que concierne a una diferencia irreductible a la oposición dialéctica, una diferencia \enquote{más profunda} que una contradicción (\emph{Diferencia y repetición}), una diferencia en la afirmación felizmente repetida (\enquote{sí, sí}), la asunción del simulacro, Deleuze sigue siendo sin duda, a pesar de tantas diferencias, aquel de quien me he considerado siempre más cerca de entre todos los de esta \enquote{generación}, jamás he sentido la menor \enquote{objeción} insinuarse en mí.\footcite{@6994-DERRIDA2005}
\end{quote}

Esta cercanía en la que no surge la menor objeción, no debe dejar de evidenciar una serie de distanciamientos irreductibles. Si ambos piensan la diferencia trascendental, la cuestión es cómo la piensan. Aquí quisiéramos señalar cuál es a nuestro criterio la diferencia central entre sus planteos y aquello que queda por pensar allí.\footcites[Quizá otra distancia surja de las formas de abordar la tradición, en la tensión entre una lectura crítica minuciosa de la tradición y una apuesta a la creación de algo nuevo: \enquote{(\dots) pues no se trata solamente de apuntar, a causa de la indecisión, a la detención de las proposiciones del discurso metafísico, sino sobre todo a la invención de un discurso diferente que fluye y abre en otro lugar otras formas de pensar y de sentir}.][114]{@6985-MENGUE2008}

Esta diferencia se encuentra en el estatuto de la negatividad en la diferencia en cuanto tal\index[concepto]{Diferencia!en cuanto tal}, esto es, un distanciamiento en sus respectivas herencias de Hegel. Si bien ambos coinciden en la crítica a Hegel, el lugar desde el cual lo realizan es diferente. En el caso de Deleuze, todo el esfuerzo teórico está dirigido a pensar una diferencia previa a la negatividad, y así una diferencia afirmativa. Esto debido a las consecuencias éticas de la dialéctica como afirmación del trabajo, el dolor y en última instancia negación del placer. Al mismo tiempo que la dialéctica sólo puede pensar la diferencia como oposición y con ello la composibilitación de fuerzas surge siempre de un choque donde las fuerzas se limitan entre sí. Para Deleuze pensar la diferencia como oposición o contradicción conlleva, de un lado, la reducción de la diferencia a la identidad del concepto y, de otro lado, una limitación de la misma diferenciación, pues las fuerzas como opuestas no pueden sino chocar unas con otras y limitar respectivamente su despliegue: una fuerza limita otra fuerza. Esto implicaría ya no sólo un problema ético al reducir el placer, afirmar el trabajo y el dolor, sino un problema político, pues la diferencia como contradicción es un pensamiento que coacciona el desenvolvimiento diferencial (la composibilitación de la potencia) desde la contraposición de fuerzas.

En el caso de Derrida, no se trata del abandono del hegelianismo, sino de su extralimitación. La diferencia es negativa en cuanto resulta irreductible como \enquote{diferencia de\ldots}, conserva una estructura de remisión (la \enquote{ferencia} de la diferencia) que hace que cada elemento esté marcado por otros elementos. Esto será nombrado por Derrida \enquote{huella}, si cada elemento conserva la huella del resto de los elementos es porque la determinación nunca puede ser unilateral, de ahí que la negatividad resulte irreductible. Negatividad que Derrida, en una lectura rigurosa de Bataille, escapa a su definición hegeliana. Tal como hemos señalado, la negatividad hegeliana es interpretada como una economía del sentido donde cada negación adquiere sentido desde la lógica de la totalidad, es decir, la dialéctica es un movimiento de reapropiación de cada elemento diferencial en un todo reconciliado consigo mismo. Frente a ello, Bataille abre la posibilidad de pensar una negatividad sin empleo donde se vuelve imposible la apropiación hegeliana, donde se rompe cualquier lógica que le otorgue sentido a los elementos en juego. De hecho la idea de \enquote{gasto sin reserva}, aquellos gastos excesivos de energía como el erotismo, la poesía o el sacrificio, suspenden cualquier lógica del intercambio económica. De allí que Derrida hable de un hegelianismo sin reserva, de una negatividad que destituye la misma posibilidad de la totalidad. Esto resulta imposible porque la alteridad sólo es posible desde múltiples mediaciones.

En resumidas cuentas, si en Deleuze la inmanencia es el lugar por excelencia de la diferencia, en Derrida es su ruptura en tanto la negatividad sin círculo hace del ser algo constitutivamente equívoco. \footcites[Nancy señala que la distancia podría encontrarse entre las formulas: \enquote{diferir consigo mismo} y \enquote{sí mismo difiriéndose}, de Deleuze y Derrida respectivamente. En el primer caso es el ser univoco que se dice de las diferencias, en el segundo caso, el ser nunca puede decirse como tal puesto que siempre difiere, no se hace presente. Si en un caso el sentido se produce en la diferencia, en el otro el sentido se vuelve imposible: \enquote{(\dots) por un lado, el sentido se apoya en la autoridad de la diferenciación, por el otro, el sentido se anula en ella. Uno hace caer todo el peso sobre el sentido como movimiento, como producción, como novedad, como devenir, el otro hace caer un peso equivalente sobre el sentido como idealidad, como identidad localizable, como verdad presentable}.][256]{@6986-NANCY2008}[Es la distinción entre la producción de lo nuevo y una suplencia de lo antiguo siempre perdido. Al mismo tiempo, Giorgio Agamben, en un texto que trabaja la inmanencia deleuzeana, establece esta oposición entre Levinas/Derrida del lado de lo trascendental y Deleuze Foucault del lado de la inmanencia.][]{@6987-AGAMBEN2008} Incluso es posible afirmar que en gran medida el esfuerzo de Derrida, más cercano a la noción de lo trascendental, es mostrar la imposibilidad de una totalidad inmanente desde que la diferencia se entiendo como un proceso infinito (aquí se comprende también como la temprana crítica a Levinas es realizada desde una recuperación de la negatividad o mediación necesaria en todo vínculo con la alteridad).\footcite{@6989-DERRIDA1989}

En el caso de Deleuze, la diferencia afirmativa dará lugar a una política de la potencia como composibilitación. La potencia, partiendo de la secundariedad de lo negativo, es \emph{a priori} composición con otras potencias. O mejor, los grados de intensidad sólo en segundo término disminuyen una potencia que es ante todo despliegue. En el caso de Derrida, la negatividad no puede ser eliminada porque se trata del mismo movimiento de la diferencia, por lo que resulta necesario pensar una negatividad sin regulación, sin empleo. Toda diferenciación es al mismo tiempo una composición que expande y limita. La diferencia hace imposible la pura inmanencia, lo que dará lugar a una política no de la potencia, sino de la comunidad (el ser de la comunidad como su propia ausencia). Potencia y comunidad nombran dos de las derivas políticas de las ontologías de la diferencia. Y con ello una serie de autores que pueden ser ubicados en una u otra herencia: Negri, Virno, para Deleuze; Nancy, Esposito, para Derrida.

Frente a estas herencias quisiéramos sugerir, sólo sugerir, que es posible otra ontología política de la diferencia a partir de la noción de diferencia trascendental. Si de un lado tenemos ciertos planteos que acentúan el carácter afirmativo de la potencia en una herencia spinoziana, de otro lado tenemos autores que acentúan el carácter negativo de la comunidad, es decir, la comunidad sin fundamento sería aquella donde no existe un rasgo \enquote{común} que de origen o sentido a la comunidad. En ambos casos, parece eliminarse la posibilidad de pensar la diferencia como tal, es decir, como proceso de diferenciación. No se trata ni de la potencia de lo dado, ni de un lugar ausente de la comunidad, sino de la serie de diferencias que configuran lo dado. Diferenciación que se entiende como condición de existencia, esto es: una configuración singular de tiempo y espacio. En otros términos, la diferencia se entiende como los procesos de temporalización y espaciamiento en tanto condiciones de existencia de lo dado. Resta pensar si en esa diferenciación tiene lugar la negatividad, no entendida como fundamento ausente, ni siquiera como la nada de lo ente, sino como la serie de mediaciones no dialécticas. O, como hemos visto, si la diferencia es una distinción unilateral o si esa unilateralidad supone ya una serie de diferenciaciones. La pregunta por el lugar de la negatividad en la diferencia queda abierta.

\section{Aperturas}

Hemos intentado a lo largo del texto indagar dos cosas: de un lado, en qué sentido la diferencia es ontológica, de allí que acentuamos su carácter genético en los diferentes autores; de otro lado, cuáles son los distanciamientos entre las concepciones de diferencia. Esto en vistas a pensar cómo se entiende la politicidad de estas ontologías diferenciales y cuáles son las posiciones que surgen allí. Si el recorrido por lecturas de la diferencia nos ha posibilitado destacar su carácter constitutivo, su carácter político surge de la existencia de un \emph{polemos} en el mismo proceso de diferenciación.

En un primer momento en una presentación contextual señalamos que el pensamiento de la diferencia surge en oposición a la dialéctica hegeliana. Discusión central en tanto la diferencia es inherente al movimiento dialéctico, incluso el movimiento de la negatividad puede ser pensado como diferenciación. De allí la necesidad de mostrar la distancia con este planteo, tal como surge en Heidegger y Saussure. En ambos casos se trata de romper con una concepción de la diferencia relacional. En el caso de Heidegger, la radicalización de la diferencia ontológica permite pensar la diferencia en cuanto tal\index[concepto]{Diferencia!en cuanto tal} como lo que hace posible la diferencia entre ser y ente. En el caso del estructuralismo, se muestra que la diferencia es lo que le otorga valor o identidad a cada signo.

En un segundo momento, presentamos la diferencia en Gilles Deleuze calificándola de afirmativa. Desde el trabajo de lectura de Nietzsche, la diferencia se distancia de la negatividad hegeliana. Se trata de pensar, para el autor, una diferencia sin negatividad. Esto implica liberar a la diferencia de su apropiación por la identidad desde un esquema representativo. La representación reduce la diferencia al subordinarla a la identidad del concepto. Frente a ello un pensamiento de la diferencia interna lleva a la cuestión de la determinación de lo dado como distinción unilateral, lo que se comprende desde la expresión \enquote{hacer la diferencia}. Pero para evitar la asimilación al plano de lo dado, Deleuze aporta la noción de diferenciación que se mueve entre el plano de lo virtual y lo actual. La diferencia es diferencia individuante.

En un tercer momento, presentamos la diferencia en Jacques Derrida calificándola de negativa. Desde una primera referencia en la lectura temprana de Husserl, la diferencia surge en un cuestionamiento al fonologismo de la tradición. Por este motivo el neologismo \emph{différance} resulta solo legible y no audible. Los antecedentes de Nietzsche, Freud y Heidegger son centrales para entender el triple estatuto de la diferencia en Derrida: diferencia como espaciamiento, esto es, diacriticidad; diferencia como temporalización, esto es, retraso originario; diferencia como \emph{polemos}, esto es, economía de la violencia. En los planteos del autor la referencia a Hegel, y específicamente a Bataille, es importante porque allí se indica que no se trata de abandonar la dialéctica, sino de extralimitarla. Pensar la diferencia desde una negatividad sin empleo.

En un cuarto momento, procedimos en un doble movimiento para mostrar las implicancias de una ontología política diferencial. Primero, señalando que los planteos de Deleuze y Derrida dan lugar a la diferencia como \enquote{constitución} de lo dado. Para ello hemos propuesto una esquema donde la diferencia se complejiza desde los calificativos óntica, ontológica y trascendental. Se trata de pensar la política como el proceso de diferenciación en tanto condición de existencia de lo dado. Segundo, hemos destacado que la distancia entre los autores puede ubicarse en su relación con la negatividad. Desde este distanciamiento señalamos que dos herencias posibles en el pensamiento contemporáneo acentúan, por caso, el carácter afirmativo de la potencia y el carácter negativo de la comunidad. Frente a ello, pensar la diferencia en su carácter trascendental permite dar cuenta de una instancia de diferenciación activa, y con ello la espacialización y la temporalización como sus caracteres fundamentales. En este sentido, en ningún caso una ontología diferencial es una ingenua afirmación de la pluralidad de lo existente, como si diferencia mentara una especie de pluralismo social. Por el contrario, el estatuto trascendental de la diferencia no sólo no se identifica con el pluralismo empírico sino que es la posibilidad de su cuestionamiento.

Indicaciones que tienen el lugar de aperturas a indagar. Y esto porque, frente a la formulación de conclusiones que cierren un texto, un pensamiento de la diferencia es un habitar constante con cierta incomodidad. Lo que da lugar, también, a una cierta política del pensamiento: \enquote{Lo que comparten [Deleuze y Derrida] es también esto: que filosofar es entrar en la diferencia, salir de la identidad y, en consecuencia, tomar las medidas y asumir los riesgos que tal salida exige. Acaso de eso se trata desde el comienzo de la filosofía: de no poder quedarnos quietos ahí donde en principio nos parece estas puestos, seguros de un suelo, de una morada y de una historia}.\footcite[21]{@6986-NANCY2008}

\section*{Referencias}
\printbibliography[heading=none]


\ifPDF
\separata{Ontología de la diferencia}
\fi
